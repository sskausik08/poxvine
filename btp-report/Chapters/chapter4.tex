\chapter{System Design}
POXVine consists of three main components.
\begin{itemize}
	\item The \emph{host mapper} is responsible to map the virtual network entities (hosts and switches) onto the physical topology. This mapping can be done based on different heuristics, so POXVine allows you to customize the host mapper. We have developed a host mapper \emph{MinSwitchMapper}, which tries to minimize the number of \emph{physical switches} which contain rules to the virtual topology.
	
	\item The \emph{network Mapper} is an application built over the \emph{POX} controller which uses the \emph{virtual-to-physical} mappings to add the required routing OpenFlow rules on the mininet switches, so that the virtual hosts can talk to one other. Another important design consideration is that the \emph{virtual network abstraction} must be preserved, that is, if a packet is to flow across a route in the virtual topology, on the physical topology, it must traverse the virtual network entities in the same order.
	
	\item The \emph{Mininet} infrastructure is used to emulate the physical network topology and the virtual hosts which are connected to the emulated physical switches (according to the \emph{virtual-to-physical}) mappings). 	
\end{itemize}  
I explain the individual components in the coming sections.


\begin{figure}
	\noindent
	\makebox[\textwidth]{\includegraphics[width=17cm]{Figures/poxvine.png}}%
	\caption{POXVine Architecture}
\end{figure}

\section{MinSwitchMapper}
The host mapper module is responsible for finding the \emph{virtual-to-physical} mappings of the virtual network entities, i.e on which physical hosts, the virtual hosts and switches are placed. This mapping can be done according to various considerations, like \emph{maximizing number of virtual hosts, minimizing the number of switches mapped to a virtual topology, greedy host allocation, providing bandwidth guarantees etc.} \\

All the virtual network entities are mapped to physical hosts, which are connected by the physical network topology. Consider the network graph which is formed by using only the switches and links that are required to connect all the physical hosts (we use the shortest path between two hosts in the network graph). Figure 2.2 demonstrates an example of such a graph. \\
\begin{figure}
	\noindent
	\makebox[\textwidth]{\includegraphics[width=17cm]{Figures/networkg.png}}%
	\caption{Example of the Network Graph Heuristic. (A) shows the physical network topology. Suppose if the virtual hosts are mapped to the hosts as shown in (A). The network graph consisting of shortest paths to all the hosts are shown in (B)}
\end{figure}

We have developed \emph{MinSwitchMapper}, which minimises the diameter of the network graph connecting the hosts of the virtual topology, The basis for this heuristic is that the traffic of this tenant's hosts are confined to the \emph{smallest portion} in the physical topology. This also minimises the number of switches where rules regarding this virtual topology is installed, thus increasing the number of tenants we can accomodate in POXVine (provided the physical host capacity is not insufficient)

\section{NetworkMapper}
The \emph{NetworkMapper} module is an application built on top of the POX controller. The role of the NetworkMapper is to use the \emph{virtual-to-physical mappings} generated by the host mapper module and add the required routing rules on the mininet switches for the virtual hosts of a tenant. One important design decision is that the NetworkMapper preserves the \emph{virtual network abstraction}. Let us suppose there are two virtual hosts \emph{v1} and \emph{v2}, connected by a path of two switches \emph{vs1} and \emph{vs2}, i.e $v1 \rightarrow vs1 \rightarrow vs2 \rightarrow v2 $.  Irrespective of the mapping, the Network Mapper must add the rules such that traffic from $v1 \rightarrow v2 $ must traverse through $vs1$, $vs2 $ and $v2$ in that order. 

\subsection{Route Tagging}
\begin{figure}
	\noindent
	\makebox[\textwidth]{\includegraphics[width=12cm]{Figures/rt.png}}%
	\caption{Mapping of \emph{v1, v2, vs1 and vs2}. (1),(2) and (3) depict the three different rules a packet from $v1$ to $v2$ needs at switch $s3$.}
\end{figure}
Consider the two virtual hosts \emph{v1} and \emph{v2}, connected by the following path  in the virtual topology.
\begin{center}
	$v1 \rightarrow vs1 \rightarrow vs2 \rightarrow v2 $
\end{center}

Consider the mapping as shown in Figure 2.3. The path taken by a packet from $v1$ to $v2$ must go to $vs1$, then $vs2$ then $v2$. At switch s3, we need three different rules for this packet. 
\begin{enumerate}
	\item The packet from $v1$ reaches $s3$ for the first time. The packet is sent out to $vs1$. 
	\item The packet is received from $vs1$. The packet is sent to out to $s2$ and will be subsequently sent to $vs2$
	\item The packet is received after traversing $vs2$. The packet is sent out to $v2$.
\end{enumerate}
Therefore, the rules added at $s3$ cannot differentiate these three kind of flows using just the IP headers. For this, we incorporate \emph{RouteTags} in the packet header \cite{simple}. In our case, the VLAN ID header field is used to store both the \emph{tenant ID} and the \emph{RouteTag}. Using the RouteTag, we can differentiate which part of the route the packet is in. Listing an example of the rules on switch $s3$. 
\begin{center}
	Rule 1 $\rightarrow$ \\
	$Match$ : IP Src=$v1$ $|$ IP Dst=$v2$ $|$ RouteTag=1 \\
	$Action$ : Output=$vs1$ $|$ RouteTag=2 \\
	Rule 2 $\rightarrow$ \\
	$Match$ : IP Src= $v1$ $|$ IP Dst= $v2$ $|$ RouteTag = 2 \\
	$Action$ : Output=$s2$ $|$ RouteTag = 3 \\
	Rule 3 $\rightarrow$ \\
	$Match$ : IP Src= $v1$ $|$ IP Dst= $v2$ $|$ RouteTag = 3 \\
	$Action$ : Output=$v2$ 
\end{center}
Thus, we identify straight paths in the network route and assign a Route Tag for each of them, thus the switch can distinguish which
part of the network route the packet is in. We will look at Route Tag calculation in the next chapter.

\subsection {Switch Tunnelling} 
As seen in the previous example, we know that $s3$ needs to have three different rules for the different Route Tag 
packets. Let us consider switch $s2$. Adding three different rules for $s2$ is wasteful, as for $s2$, the route tag is 
of no importance. It just sends packets from $s1$ to $s3$ and $s3$ to $s1$. Inspired from \cite{simple}, we establish
switch tunnels in the network.  Thus, if we divide the physical network route into \emph{segments} (where the start and end of each segment is connected to a virtual entity), then the switches in the middle of the segment do not need fine grained rules, they need to route the packet to the end-switch of the segment. \\

Thus, the NetworkMapper adds routing rules for every other switch on each switch. At the start of each \emph{segment}, the rules added modify the source MAC address (unused field for the POXVine system) to indicate the end-switch of the segment. The switches in the middle will just route the packet to that switch. Revisiting the example in Figure 2.3, switch $s1$ will have the following rule. 
\begin{center}
	$Match$ : IP Src=$v1$ $|$ IP Dst=$v2$ \\
	$Action$ : Output=$s2$ $|$ RouteTag=1 $|$ MAC Src=$s3$\\
\end{center}

Switch $s2$ will have switch tunnel rules to switch $s1$ and $s3$. 
\begin{center}
	Rule 1 $\rightarrow$\\
	$Match$ : MAC Src=$s3$ \\
	$Action$ : Output=$s3's\ Port\ Number$ \\
	Rule 2 $\rightarrow$\\
	$Match$ : MAC Src=$s1$ \\
	$Action$ : Output=$s1's\ Port\ Number$ \\	
\end{center}

\section{Mininet Infrastructure Creator}
POXVine uses Mininet to create a \emph{emulated} physical/virtual topology as per the specifications. From the topology configurations and the virtual-to-physical mappings, a hybrid topology configuration is created for the Mininet Infrastructure Creator, which comprises of all the physical topology's switches and links, and according to the virtual-to-physical mappings, the virtual hosts and switches connected to the corresponding physical switches. We do not create the physical hosts in Mininet. Future versions of POXVine can run the virtual hosts and switches on real 'physical' hosts.








 

