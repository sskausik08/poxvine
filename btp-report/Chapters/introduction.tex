\chapter{Introduction}

\section{Software Defined Networks}
Software defined Networks is an emerging architecture which provides a framework to manage network services through the abstraction of lower level functionality. In SDNs, the \emph{control plane} which makes the decision of where to send traffic is decoupled from the \emph{data plane}, which performs the actual forwarding of traffic. The advantages of SDN over traditional network architectures are listed below.
\begin{itemize}
	\item \textit{Central State} : The entire state of the network and name bindings exist in a central location, called the controller. All inputs from the network are passed to the controller, which decides the policy needs to be implemented. 
	\item \textit{Decoupled Control and Forwarding} : In SDNs, the control plane is separated from the data plane. The controller performs the route computation and push the forwarding rules to the switches. The switches perform the forwarding of packets. 
	\item \textit{Software Controller} : In SDNs, the controller is implemented in software, so can be modified to implement any kind of policy. The switches perform basic forwarding and expose a common API for the controller to talk to them. Because the control plane is in software, changes in network protocols and services are easier to implement without a major hardware overhaul. 
\end{itemize}

\section{POX}
POX \cite{pox} is a single-threaded Python-based controller. It is widely used for fast prototyping of network applications in research. At its core, it’s a platform for the rapid development and prototyping of network control software. Convenient to setup up for research experiments makes POX an excellent SDN controller to play with and develop.
  
\section{Network emulation using Mininet}
Mininet \cite{mininet} is a network emulator which creates a network of virtual hosts, switches, controllers, and links. Mininet hosts run standard Linux network software, and its switches support OpenFlow for highly flexible custom routing and Software-Defined Networking. Mininet provides a simple and inexpensive network testbed for developing OpenFlow applications, with the need of an actual physical OpenFlow enabled topology. Mininet has been accelerating research on Software Defined Networking.

\section{Organization of the Report}
The rest of the report is organized as follows: In Chapter 2, we discuss the problem description and inspiration and use cases of POXVine. We discuss related work in Chapter 3. In Chapter 4, we describe the design of POXVine system and Implementation in Chapter 5. We present some experiments using POXVine in Chapter 6 and present conclusions and future work in Chapter 7.  



