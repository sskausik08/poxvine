\chapter{Introduction}

\section{Software Defined Networks}
Software defined Networks is an emerging architecture which provides a framework to manage network services through the abstraction of lower level functionality. In SDNs, the \emph{control plane} which makes the decision of where to send traffic is decoupled from the \emph{data plane}, which performs the actual forwarding of traffic. The advantages of SDN over traditional network architectures are listed below.
\begin{itemize}
	\item \textit{Central State} : The entire state of the network and name bindings exist in a central location, called the controller. All inputs from the network are passed to the controller, which decides the policy needs to be implemented. 
	\item \textit{Decoupled Control and Forwarding} : In SDNs, the control plane is separated from the data plane. The controller performs the route computation and push the forwarding rules to the switches. The switches perform the forwarding of packets. 
	\item \textit{Software Controller} : In SDNs, the controller is implemented in software, so can be modified to implement any kind of policy. The switches perform basic forwarding and expose a common API for the controller to talk to them. Because the control plane is in software, changes in network protocols and services are easier to implement without a major hardware overhaul. 
\end{itemize}

\section{POX}

\section{Network emulation using Mininet}